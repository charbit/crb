\documentclass[a4paper, 12pt]{report}
%================================================================
\usepackage{color,amsmath,amsfonts,amssymb,epsfig,hyperref}
\usepackage{graphicx}
\usepackage{dsfont}
\usepackage[latin1]{inputenc}
\usepackage[T1]{fontenc}
\usepackage[english]{babel}
\usepackage{pdfpages}
%====================================
 \textheight 22cm
% \doublespace
  \oddsidemargin -0.5cm
  \evensidemargin +1.5cm
  \textwidth 15cm
 \topmargin 0cm
%============================================
% Figures
%============================================
\newcommand{\figsstit}[2]{
\begin{figure}[hbtp]
\centerline{
    \hbox{ \includegraphics[scale=#2]{#1} }
}
\end{figure}}
%============================================
\newcommand{\figscale}[4]{
\begin{figure}[hbtp]
\centerline{
    \hbox{ \includegraphics[scale=#4]{#1} }
}
\begin{center}
\parbox{12 cm}
{
    \caption{\protect\small\it  {#2}}
    \label {#3}
}
\end{center}

\end{figure}}

%==================================================
\newcommand\algo[1]%
{
    \begin{center} %
    \begin{tabular} {||p{10 cm}l ||}%
    \hline
               #1 &  \\
    \hline
    \end{tabular}
    \vspace{12pt}
    \end{center}
}


%==============================================
\newcommand{\prob}[1]{\mathds{P}\left( #1 \right)}
\newcommand{\esp}[1]{\mathds{E}\left[ #1 \right]}
\newcommand{\var}[1]{\mathrm{var}\left( #1 \right)}
\newcommand{\cov}[1]{\mathrm{cov}\left( #1 \right)}
\newcommand{\diag}[1]{\mathrm{diag}\left( #1 \right)}
\newcommand{\trace}[1]{\mathrm{trace}\left( #1 \right)}
\newcommand{\card}[1]{\left| #1 \right|}
\newcommand{\myemph}[1]{\emph{\color{red}#1}}

%%============================================================================
\def\thesection{\arabic{section}}
%\def\thesubsection{\arabic{section}.\arabic{subsection}}
%\def\thesubsubsection{\arabic{section}.\arabic{subsection}.\arabic{subsubsection}}
%\def\thefigure{\arabic{figure}}
%\def\theequation{\arabic{equation}}
%\def\theexercice{\arabic{exercice}}
%\def\theexample{\arabic{example}}
%\def\theproof{\arabic{proof}}

%===============================================
\newtheorem{property}{Properties}
\newtheorem{remark}{Remark}
\newtheorem{theorem}{Theorem - \thetheoreme}
\newtheorem{definition}{Definition - \thedefinision}
\newtheorem{example}{Example}
\newtheorem{lemme}{Lemme - \thelemme}
\newtheorem{proof}{Proof - \theproof}
\newenvironment{TAB}{\begin{table}[[hbt] \center \leavevmode}{\end{table}}
%%============================================================================
%\renewcommand\arraystretch{1.6}

\def\ua{\underline a}
\def\ub{\underline b}
\def\uB{\underline B}
\def\uH{\underline H}
\def\ur{\underline r}
\def\us{\underline s}
\def\ux{\underline x}
\def\uX{\underline X}
\def\uZ{\underline Z}
\def\utheta{\underline \theta}



\def\tn{\mathrm{TN}}
\def\fn{\mathrm{FN}}
\def\tp{\mathrm{TP}}
\def\fp{\mathrm{FP}}
\def\tpn{\mathrm{tPN}}
\def\tnn{\mathrm{tNN}}
\def\tdn{\mathrm{tDN}}

\def\precision{\mathrm{\color{red}Precision}}
\def\recall{\mathrm{\color{red}Recall}}
\def\fscore{{\color{red}F\mathrm{-score}}}
\def\far{\mathrm{FAR}}
\def\mdr{\mathrm{MDR}}
\def\vdr{\mathrm{VDR}}
\def\ci{\mathrm{CI}}
\def\pfa{P_{\mathrm{FA}}}
\def\pd{P_{\mathrm{D}}}
\def\loc{\mathrm{LOC}}

\def\SNR{\mathrm{SNR}}
\def\crb{\mathrm{CRB}}
\def\fim{\mathrm{FIM}}

\def\auc{\mathrm{AUC}}

\def\void{{\small void}}
\def\nomeaning{{\small meaningless}}
\def\unknown{{\small unknown}}
\def\MSC{\mathrm{MSC}}
\def\hMSC{\widehat{\MSC}}%{\MSC}} 
\def\ellk{{k}}
\def\SOI{common signal part }
\def\absGamma{\Phi}

%============== colors ========================
\definecolor{enstrouge}{RGB}{212,65,84}
\definecolor{lightorange}{RGB}{235,226,52}
\definecolor{greennoise}{RGB}{243,42,255}
\definecolor{lightred}{RGB}{255,181,183}
\definecolor{light-grey}{rgb}{0.95,0.95,0.95}
\definecolor{peach}{rgb}{0.98,0.49,0.25}
\definecolor{burntorange}{rgb}{0.79,0.37,0}
\definecolor{light-yellow}{rgb}{1,1,0.92}

\definecolor{light-green}{RGB}{231,255,145}
\definecolor{enstorange}{RGB}{255,214,10}
\definecolor{enstrouge}{RGB}{212,65,84}
\definecolor{grey}{RGB}{204,204,204}
\definecolor{blue}{RGB}{0,0,255}
\definecolor{almost-black}{RGB}{100,100,100}
\definecolor{violet}{rgb}{0.4,0,0.4}
\definecolor{cyan}{RGB}{0,255,255}
\definecolor{magenta}{RGB}{243,42,255}

\def\degree{^{\circ}}
\def\simiid{\stackrel{\mathrm{i.i.d.}}{\sim}}
\def\simind{\stackrel{\mathrm{ind.}}{\sim}}

 
%%%============================================================================
%%\def\thesection{\arabic{section}}
%%\def\thefigure{\arabic{figure}}
%%\def\theequation{\arabic{equation}}
%%\def\theexercice{\arabic{exercice}}
%%\def\theequation{\arabic{exercice}.\arabic{equation}}
%%%============================================================================
%%\newcounter{auxiliaire}
%%%%%%%%% comment
%%\setcounter{auxiliaire}{\theenumi}
%%\end{enumerate}
%% TEXTE
%%\begin{enumerate}
%%\setcounter{enumi}{\theauxiliaire}
%%%============================================================================

 \bibliographystyle{plain} 

\begin{document}
 \sloppy
%=======================================================
%=======================================================


In this study we are interested by the station design in term of parameter accuracy. In short, in presence of only pure coherent acoustic waves, it is clear that the best design is to locate sensors at far as possible. For which reason could we take into account an upper bound on the aperture ? Regarding the size of the station and the distance of the source, the only parameter which can induce a limitation on the previous design is the possible loss of coherence on distant pairs of sensors, in presence of coherent signal. By definition, two signals arriving on a two different locations are said non-coherent signals, or simply noises, if they are uncorrelated.

Therefore to determine a good design, it is necessary to take into account the trade-off between the distance and the possible loss of coherence. A very naive model consists to take:
\begin{eqnarray*}
 \loc(f) &=& e^{-\beta \frac{d}{\lambda}}%e^{-\alpha df/c}
\end{eqnarray*}
where $\lambda$ denotes the wavelength and $d$ the distance between 2 sensors. If $\beta=0$ there is no loss of coherence. This very simple model has to be validated but here we assume that it is.

\begin{remark}[on the noise]
Typical station aperture  is 2 km. Noise is mainly due to the wind.
\begin{itemize}
\item
Therefore we assume that the noises are spatially uncorrelated regarding the sensor inter-distances.
\item
On the other hand, we assume that the noise levels are identical on all sensors. Although that is not realistic, it is worth to notice that the noise level is not directly related to the inter-distances between sensors. It follows that, for the station design understanding, we can consider there is no loss of generalities to assume that.
\end{itemize}
\end{remark}

\begin{remark}[on the coherent source]
To be able to study the LOC, we need a permanent source. The microbarom plays this role in the following.
\end{remark}

\begin{remark}[on the isotropy]
In the absence of LOC, the station is isotrope iff 
\begin{eqnarray}
 \label{eq:isotropy-condition}
XX^{T}&\propto&I_{d}
\end{eqnarray}
where $X$ is $d\times M$ matrix whose the $m$-th colomun if the coordinates of the $m$-th sensor (in any  system of coordinates). 

It follows that  given an arbitrary array with $M$ sensors in $\mathds{R}^{d}$ it is always possible to complete with $d$ locations in such a way the new array is isotrope, i.e. verifies \eqref{eq:isotropy-condition}, see annex.

\end{remark}

\begin{remark}[on the geometry]
In the absence of LOC, we will see below that the best performances are associated with inter-distances as large as possible. In this case the maximization of the minimal distance leads to locate the sensors on a circle. But in this case the distribution of the distances is not uniform. It follows that in presence of LOC, several sensors are concern with the same LOC. It seems (it is not a proof) that a rule could be to locate the sensors in such a way that the distances are more uniform (see figure \ref{fig:uniforinterdistances}).

\figscale{../figures/uniforinterdistances.pdf}{Sensor locations with inter distances}{fig:uniforinterdistances}{0.9}

\end{remark}


\newpage\clearpage
%=======================================================
 \section{LOC model}
%=======================================================
%=======================================================


We consider a station with $M$ sensors. There is only one acoustic source faraway from the station, in such a way it can be considered as planar wave. Therefore the $M$-ary signal  writes: 
\begin{eqnarray*}
x(t) & = & \underbrace{s(t;\theta)}_{\text{acoustic signal}}+ \underbrace{w(t)}_{\text{noise}}
\end{eqnarray*}
where $\theta$ denotes the 3D slowness vector. Under assumption of pure delay we can write for the $m$-th sensor located in $r_{m}$:
\begin{eqnarray}
\label{eq:mthentryofst}
s_{m}(t;\theta)&=&s(t-r_{m}^{T}\theta)
\end{eqnarray}
It follows  that, under the assumption  that $s(t)$ is stationary random process with spectral density $\gamma_{s}(f)$, the spectral matrix of the $M$-ary process $s(t)$, whose the $m$-th entry is \eqref{eq:mthentryofst}, writes:
\begin{eqnarray*}
\Gamma_{s}(f)&=&\gamma_{s}(f)d(f)d^{H}(f)
\end{eqnarray*}
where $d(f)$ is a $M$-ary vector whose the $m$-entry writes $e^{-2j\pi fr_{m}^{T}\theta}$. Clearly the matrix $\Gamma_{s}(f)$ is of rank 1, and that is the definition of the coherence. Loss of coherence (LOC) means that $\Gamma_{s}(f)$ is of rank greater than 1.


\subsubsection{Discrete domain}

All signals are real and sampled at the sampling frequency  $F_{s}=20$ Hz. After sampling we obtain $x_{n}=x(n/F_{s})$. For $k=0$ to $(N-1)$ we consider the $M$-ary discrete Fourier transform:
 \begin{eqnarray*}
 X_{k}&=&\frac{1}{\sqrt{N}}\sum_{n=0}^{N-1}x_{n}\,e^{-2j\pi n f_{k}}
    \quad\mathrm{where}\quad
 f_{k}=kF_{s}/N
 \end{eqnarray*}
We let $K=N/2$.
For $K$ great enough, $X_{1}$, $\ldots$, $X_{K}$ is a sequence of  $M$-ary independent circularly gaussian random vectors with zero-mean and respective covariance:
\begin{eqnarray}
\label{eq:spectralmatrixpuredelay}
\Gamma_{k}(\alpha)&=&\gamma_{k}d_{k}(\theta)d_{k}^{H}(\theta)+\sigma^{2}I_{M}
%\diag{\begin{matrix}\sigma_{1},\ldots,\sigma_{M}\end{matrix}}
\end{eqnarray}
where $d_{k}(\theta)$ is an $M$-ary vector whose the $m$-th entry writes $D_{k,m}=e^{-2j\pi f_{k} r_{m}^{T}\theta }$ and where $\gamma_{k}=\gamma_{s}(f_{k})$.

The expression \eqref{eq:spectralmatrixpuredelay} can be rewritten:
\begin{eqnarray}
\label{eq:spectralmatrixLOC}
\Gamma_{k}(\alpha)&=&\gamma_{k}D_{k}(\theta)\, C_{k}\,D_{k}^{H}(\theta)+\sigma^{2}I_{M}
%\diag{\begin{matrix}\sigma_{1},\ldots,\sigma_{M}\end{matrix}}
\end{eqnarray}
where $D_{k}(\theta)$ is an $M$-ary diagonal matrix whose the $m$-th diagonal entry writes $D_{k,m}=e^{-2j\pi f_{k} r_{m}^{T}\theta }$ and where $C_{k}=\mathds{1}\mathds{1}^{T}$ which is a rank 1 matrix.


Under LOC, $C_{k}$ is no more a rank 1 projector.  In the following to take into account the LOC we consider that 
\begin{eqnarray*}
 C_{k,\ell\ell'} (\beta)&=&e^{-\beta \frac{d_{\ell,\ell'}}{\lambda_{k}}}=e^{-\beta \frac{d_{\ell,\ell'}f_{k}}{c}}
\end{eqnarray*}
where $\lambda_{k}=c/f_{k}$ is the wavelength, $d_{\ell,\ell'}$ the distance between 2 locations $\ell$ and $\ell'$ in the 3D space, and $\beta$ a positive constant. For $\beta=0$, $C(\beta)=\mathds{1}\mathds{1}^{T}$ leading to a perfect coherence. If $\beta\approx+\infty$, $C(\beta)=I_{M}$ and there is no coherence between any 2 locations.

Summarizing   we can write that:
\begin{eqnarray*}
 (X_{1},\ldots,X_{K}) &\sim&\prod_{k=1}^{K}\mathcal{N}_{c}(x_{k};0_{M},\Gamma_{k}(\alpha))
\end{eqnarray*}
and the likelihood writes:
\begin{eqnarray}
 \mathcal{L}(\alpha)&=&
 \sum_{k=1}^{K}\log\det\Gamma_{k}(\alpha)+\trace{\Gamma_{k}^{-1}(\alpha)X_{k}X_{k}^{T}}
\end{eqnarray}
where the parameter $\alpha$ consists of
\begin{eqnarray}
\alpha&=&
\{
\gamma_{1},\ldots,\gamma_{K},\theta_{1},\theta_{2},\theta_{3},\beta,\sigma^{2}
\}
\in \mathds{R}^{+K}\times  \mathds{R}^{3}\times \mathds{R}^{+}\times \mathds{R}^{+}
\end{eqnarray}
Its size is $K+5$.

\begin{remark}
To avoid intractable computation, we assume in the following that $\beta$ is known, leading to an uncertainty on only $\theta_{1}$, $\theta_{2}$, $\theta_{3}$, $\sigma^{2}$, $\gamma_{1}$, $\ldots$, $\gamma_{K}$. Now the unknown parameter has dimension $K+4$.
\end{remark}
%=======================================================
 \section{Cramer-Rao}
%=======================================================
%=======================================================
The $(\ell,\ell')$ entry of the Fisher information matrix  associated to the parameter $\alpha$ writes:
\begin{eqnarray}
 \label{eq:FIM}
\fim_{\ell,\ell'}(\alpha)&=&\sum_{k=1}^{K}\trace{\Gamma_{k}^{-1}\times\partial_{\ell}\Gamma_{k}\times\Gamma_{k}^{-1}\times\partial_{\ell'}\Gamma_{k}}
\end{eqnarray}
where $\ell,\ell'\in\{1,2,3,4\}$ and where $\partial_{\ell}\Gamma_{k}$ is the partial derivative w.r.t. $\theta_{1}$, $\theta_{2}$, $\theta_{3}$ and $\sigma^{2}$.  It is clear that the CRB terms relative to the correlation of $\theta$ and $\sigma^{2}$ are 0. We let:
\begin{eqnarray*}
 d_{k,\ell}(\theta)&=&
  \begin{bmatrix}
  -2j\pi f_{k} r_{1,\ell}\,e^{-2j\pi f_{k}r_{1}^{T}\theta}
  \\
  \vdots
  \\
  -2j\pi f_{k} r_{M,\ell}\,e^{-2j\pi f_{k}r_{M}^{T}\theta}
  \end{bmatrix}
\end{eqnarray*}
Then for $\ell=1,2,3$:
\begin{eqnarray*}
\partial_{\ell}\Gamma_{k}&=&\gamma_{k}
   \diag{d_{k,\ell}(\theta)}C_{k}(\beta)D_{k}^{H}(\theta)+
   D_{k}(\theta)C_{k}(\beta) \diag{d_{k,\ell}(\theta)}^{H}
   \\
   &=&2\, \mathcal{R}\left(
   \diag{d_{k,\ell}(\theta)}C_{k}(\beta)D_{k}^{H}(\theta)\right)
\end{eqnarray*}
It is worth to notice that, if $\beta\approx +\infty$, $C_{k}=I_{M}$ and $\partial_{\ell}\Gamma_{k}=0$ which is normal because in this case $\Gamma_{k}$ does not depend on $\theta$. For the derivation w.r.t. $\sigma^{2}$ we have:
\begin{eqnarray*}
\partial_{4}\Gamma_{k}&=&I_{M}
\end{eqnarray*}
A direct consequence is that the FIM is of this shape:
\begin{eqnarray*}
 \fim&=&\begin{bmatrix}
 F_{123}&0_{3,1}&0_{3,K}
 \\
 0_{1,3}&f_{\sigma^{2}}&?
 \\
 0_{K,3}&?&F_{\gamma,K,K}
 \end{bmatrix}
\end{eqnarray*}
Because the CRB on the estimation of $\theta$ is given by the $3\times 3$ matrix of the inverse of $\fim$ and because the correlations with other parameter estimates are 0, we are only concern with $ F_{123}$ by taking $ F_{123}^{-1}$ . Then the CRB w.r.t. the azimuth, elevation and velocity can be derived using the Jacobian of the one-to-one mapping between the slowness and the vector $(a,e,c)$ where $a$, $e$ and $c$ denote respectively the azimut, the elevation and the velocity. The same calculation can be used to derive the CRB w.r.t. the azimuth and the trace velocity.


%=======================================================
%=======================================================
\appendix
\chapter{Transform an array in isotrope array}
We consider an arbitrary  array whose locations are given by:
\begin{eqnarray*}
X&=&\begin{bmatrix}
x_{1,1}&\ldots&x_{1,M}
\\
x_{2,1}&\ldots&x_{2,M}
\\
x_{3,1}&\ldots&x_{3,M}
\end{bmatrix}
\end{eqnarray*}
It follows that
\begin{eqnarray*}
 XX^{T}&=&
 \sum_{i=1}^{d}\mu_{i}\xi_{i}\xi^{T}_{i}
\end{eqnarray*}
where $0\leq \mu_{i}\leq \alpha_{0}$. The new array writes
\begin{eqnarray*}
Y &=& \begin{bmatrix}
X&\sqrt{(\alpha_{0}-\mu_{1})}\xi_{1}&\sqrt{(\alpha_{0}-\mu_{2})}\xi_{2}&\sqrt{(\alpha_{0}-\mu_{3})}\xi_{3}
\end{bmatrix}
\end{eqnarray*}
It is easy to verify that $YY^{T}=\alpha_{0}I_{d}$.
\end{document}

